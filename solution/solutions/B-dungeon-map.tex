\section{B. 비숍 여행}

\begin{frame} % No title at first slide
    \sectiontitle{B}{비숍 여행}
    \sectionmeta{
        \texttt{implementation, math}\\
        출제진 의도 -- \textbf{\color{acbronze}Easy}
    }
    \begin{itemize}
        \item 제출 0번, 정답 0명 (정답률 0\%)
        \item 처음 푼 사람: \textbf{} 
        \item 출제자: \texttt{riroan}
    \end{itemize}
\end{frame}

\begin{frame}{\textbf{B}. 비숍 여행}
    \begin{itemize}
    	\item 비숍은 한번 이동할 때마다 x좌표와 y좌표의 홀짝이 각각 변합니다.
    	\item 예를들어 비숍이 (짝수, 홀수)좌표에 있다면 한번 이동했을 때 (홀수, 짝수)좌표로 이동합니다.
    	\item 하지만 이동을 하더라도 \textbf{x좌표 + y좌표}의 홀짝은 변하지 않습니다.
    	\item 따라서 비숍의 시작좌표 합의 홀짝과 같은 동전좌표 합의 홀짝인 개수를 구하면 됩니다.
    	\item 총 시간복잡도는 \complexity{N}입니다.
    \end{itemize}
\end{frame}