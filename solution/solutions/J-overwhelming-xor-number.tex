\section{J. 압도적 XOR수}

\begin{frame} % No title at first slide
    \sectiontitle{J}{압도적 XOR수}
    \sectionmeta{
        \texttt{math, bit\_mask}\\
        출제진 의도 -- \textbf{\color{acgold}Challenging}
    }
    \begin{itemize}
        \item 출제자: 김태현\textsuperscript{\color{kupc-gray}\texttt{kth990303}}
    \end{itemize}
\end{frame}

\begin{frame}{\textbf{J}. 압도적 XOR수}
    \begin{itemize}
        \item 먼저, 압도적 xor 수의 정의부터 살펴봅시다.
        \item 압도적 xor 수의 정의를 파악하려면 $A$와 $B$의 상관관계에 대해 살펴볼 필요가 있습니다.
        \item $P$는 $A$보다 큰 2의 제곱수 중 가장 작은 값이므로, $P-1$은 $A$보다 크거나 같은 $2^i-1$ ($i$는 0보다 큰 정수)임을 알 수 있습니다.
    \end{itemize}
\end{frame}

\begin{frame}{\textbf{J}. 압도적 XOR수}
	\begin{itemize}
		\item $2^i-1$을 2진수로 나타내보면 $1$로만 이루어짐을 알 수 있습니다. 그러므로 xor 성질에 따라 $B = A \oplus (P-1)$에서 $B$는 $A$와 서로 완전히 다른 비트를 가지는 수임을 파악할 수 있습니다. 
		\item 다시 말해 $A$가 가지는 비트 성질을 완전히 제외(반전)시키므로 $B = P - 1 - A$입니다.
		\item 예를 들어 $A = 13 (1101_{(2)})$이면 $P-1 = 15 (1111_{(2)})$이고, $B = A \oplus (P-1)$은 $2 (0010_{(2)})$입니다. 
		\item $P$는 조건을 만족하는 값 중 가장 작은 값이므로 $B = P - 1 - A < A$을 만족합니다.
	\end{itemize}
\end{frame}


\begin{frame}{\textbf{J}. 압도적 XOR수}
	\begin{itemize}
		\item 이제 압도적 xor 수의 정의를 살펴봅시다.
		\item $A \ge B \cdot (2^K - 1)$일 때 $A$가 압도적 xor 수라고 합니다. 
		\item $A$와 $B$는 2진수일 때 같은 길이를 가지고 완전히 반전된 비트를 가진다는 점을 이용하여 둘의 상관관계를 다시 한 번 살펴봅시다.
		\item $A, B$를 2진수로 나타낼 때 같은 길이를 가지는 범위 내에서 $A$와 $B$의 차이가 가장 클 때는 $A$는 $1$로만 구성돼있을 때이고 $B$는 $0$으로만 구성돼있을 때입니다. 
		\item 이 때는 $B = 0$이므로 어떠한 길이이든 상관없이 $A$는 압도적 xor 수입니다. 다시 말해 $1, 3, 7, 15, \dots$ 과 같이 $2^i - 1$꼴은 반드시 압도적 xor 수입니다.
	\end{itemize}
\end{frame}

\begin{frame}{\textbf{J}. 압도적 XOR수}
	\begin{itemize}
		\item $A, B$를 2진수로 나타낼 때 같은 길이를 가지는 범위 내에서 $A$와 $B$의 차이가 가장 작을 때는 $A$는 맨 앞을 제외하고 $0$으로 구성돼있을 때입니다. 
		\item $2^i > (2^{i-1} + 2^{i-2} + … + 2^0)$($i$는 0보다 큰 정수)임이 성립한다는 점을 이용해봅시다.
		\item $A = 100000_{(2)}, B = 011111_{(2)}$일 때, $A$는 $B$의 $1$배보다 큽니다. 그리고 $A, B$의 맨 앞에서 2번째 비트가 정반되기 전까지는 $(1+2)$배보다 작음이 보장됩니다.
	\end{itemize}
\end{frame}

\begin{frame}{\textbf{J}. 압도적 XOR수}
	\begin{itemize}
		\item $A = 110000_{(2)}, B = 001111_{(2)}$일 때, $A$는 $B$의 $(1+2)$배보다 큽니다. 
		\item 그리고 $A, B$의 맨 앞에서 3번째 비트가 정반되기 전까지는 $(1+2+4)$배보다는 작음이 보장됩니다.
		\item 따라서 $A \ge B \cdot (2^K - 1)$이기 위해서는 $A, B$를 이진수로 나타냈을 때 맨 앞에서 $K$번째 비트까지 $A$는 $1$로, B는 $0$으로 구성돼야 합니다. 
	\end{itemize}
\end{frame}

\begin{frame}{\textbf{J}. 압도적 XOR수}
	\begin{itemize}
		\item $A, B$를 이진수로 나타냈을 때 서로 같은 길이일 경우, $A$가 특정 값 이상이면 압도적 xor수임이 보장됨을 이용해 수학 또는 이분탐색으로 \complexity{\log N} 시간복잡도에 답을 구할 수 있습니다.
		\item 수학으로 접근한다면 $[2^i, 2^{i+1}-1]$ 범위의 정수($i$는 음이 아닌 정수)에서 압도적 xor 수의 개수는 $2^{i+2}-1$개임을 이용하여 등비수열 합의 공식을 이용해 구할 수 있습니다.
		\item 이분탐색으로 접근한다면 $[2^i, 2^{i+1}-1]$ 범위의 정수($i$는 음이 아닌 정수)에서 압도적 xor 수가 특정 값 이상임이 되는 단조함수 꼴을 이룬다는 점을 이용해 \complexity{\log 2^i}에 구할 수 있습니다. 
		\item 이 과정을 $i \le \log N$이 될 때까지 반복해서 더해주면 됩니다.
		\item $[2^i, 2^{i+1}-1]$ 범위 경계값이 $2^K - 1$보다 작을 때에도 반드시 압도적 xor 수가 1개는 있다는 점을 주의합시다.
	\end{itemize}
\end{frame}