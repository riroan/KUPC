\section{E. 즐거운 XOR}

\begin{frame} % No title at first slide
    \sectiontitle{E}{즐거운 XOR}
    \sectionmeta{
        \texttt{pigeonhole\_principle, brute\_force}\\
        출제진 의도 -- \textbf{\color{acsilver}Medium}
    }
    \begin{itemize}
        \item 출제자: 김명기\textsuperscript{\color{kupc-gray}\texttt{riroan}}
    \end{itemize}
\end{frame}

\begin{frame}{\textbf{E}. 즐거운 XOR}
    \begin{itemize}
		\item $a_i$ 의 제한이 작음에 주목합니다.
		\item 비둘기집 원리에 의해 배열 크기가 $100$이 넘어가면 중복되는 원소가 적어도 하나 존재합니다.
		\item 원소의 개수를 세는 배열을 $A$라고 정의하겠습니다.
	\end{itemize}
\end{frame}

\begin{frame}{\textbf{E}. 즐거운 XOR}
	\begin{itemize}
		\item 이제 $(0,0,0), (0,0,1), \cdots , (100,100,100)$까지 모두 탐색을 하면 됩니다.
		\item $(a, b, c)$인 경우 $A[a]=A[a]-1, A[b]=A[b]-1, A[c]=A[c]-1$를 했을 때 배열 $A$에 음수가 없는 경우에만 탐색합니다.
		\item 탐색이 완료된 후 가장 큰 xor값을 출력하면 됩니다.
		\item 총 시간복잡도는 \complexity{100^3}입니다.
	\end{itemize}
\end{frame}