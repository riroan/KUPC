\section{A. 슬픈 건구스}

\begin{frame} % No title at first slide
    \sectiontitle{A}{슬픈 건구스}
    \sectionmeta{
        \texttt{number\_theory}\\
        출제진 의도 -- \textbf{\color{acbronze}Easy}
    }
    \begin{itemize}
        \item 출제자: \texttt{이동훈}
    \end{itemize}
\end{frame}

\begin{frame}{\textbf{A}. 슬픈 건구스}
    $(0, 0)$에서 $(n, m)$으로 레이저를 쏘는 경우를 생각해봅시다.

    \vspace{18pt}
    
    레이저가 진행하며 $x$좌표가 $n$번, $y$좌표가 $m$번 증가해야 합니다. 
    
    \vspace{18pt}스
    \begin{itemize}
        \item $x$축에 평행한 벽을 뚫을 때 마다 $y$좌표의 정수부분이 $1$ 증가합니다.
        \item $y$축에 평행한 벽을 뚫을 때 마다 $x$좌표의 정수부분이 $1$ 증가합니다.
        \item 건물을 뚫을 때 마다 $x$좌표와 $y$좌표의 정수부분이 동시에 $1$ 증가합니다.
    \end{itemize}

\end{frame}