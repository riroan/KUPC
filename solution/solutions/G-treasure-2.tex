\section{G. 보물찾기 2}

\begin{frame} % No title at first slide
    \sectiontitle{G}{보물찾기 2}
    \sectionmeta{
        \texttt{dijkstra, 0\_1\_bfs}\\
        출제진 의도 -- \textbf{\color{acgold}Hard}
    }
    \begin{itemize}
        \item 출제자: 이동훈\textsuperscript{\color{kupc-gray}\texttt{aru0504}}
    \end{itemize}
\end{frame}

\begin{frame}{\textbf{G}. 보물찾기 2}
	\begin{itemize}
		\item 시작점부터 끝점까지의 최단거리를 구하는 문제입니다.
		\item 가중치가 일정하지 않아서, 단순한 bfs를 적용할 수 없습니다.
		\item 가중치는 총 0, 1로 두 가지입니다. 이를 이용할 수 없을까요?
	\end{itemize}
	
\end{frame}

\begin{frame}{\textbf{G}. 보물찾기 2}
	\begin{itemize}
		\item 풀이 방법 중 하나는 데이크스트라\textsubscript{\texttt{\color{kupc-gray}dijkstra}}입니다.
		\item 음이 아닌 정수를 가중치로 가지므로, 해당 알고리즘을 사용해서 최단거리를 구할 수 있습니다.
		\item 이보다 더 간단하고 효율적인 방법이 있을까요?
	\end{itemize}
\end{frame}

\begin{frame}{\textbf{G}. 보물찾기 2}
	\begin{itemize}
		\item 가중치는 $0$ 또는 $1$로 두 가지입니다. 간단한 최단 거리 bfs를 생각해 봅시다.
		\item 큐에서 정점을 뽑았을 때, 해당 정점은 큐에 있는 가중치보다 작거나 같습니다.
		\item 이 성질을 활용해서, 덱\textsubscript{\texttt{\color{kupc-gray}dequeue}}을 사용합시다.
	\end{itemize}
\end{frame}

\begin{frame}{\textbf{G}. 보물찾기 2}
	\begin{itemize}
		\item 시작점부터 탐색합시다. 현재 간선의 가중치가 $0$인 경우에는 덱의 앞쪽에, $1$인 경우에는 덱의 뒤쪽에 삽입합니다.
		\item 탐색하는 동안 덱 내부 원소들의 가중치의 종류는 최대 두 가지이며, 가중치를 기준으로 정렬돼 있습니다.
		\item 우선순위 큐\textsubscript{\texttt{\color{kupc-gray}priority-queue}}의 경우, 원소를 뽑는 데 \complexity{\log{N}}이 들지만, 덱을 사용하면 \complexity{1}만에 할 수 있습니다.
	\end{itemize}
\end{frame}


\begin{frame}{\textbf{G}. 보물찾기 2}
	\begin{itemize}
		\item 이와 같은 풀이 방법을 \texttt{0-1 bfs}라고 합니다. 주로 가중치의 종류가 $0$ 또는 $1$로 두 가지뿐인 경우에 사용합니다.
		\item 데이크스트라와 구현이 비슷합니다. 정점이 $HW$개, 간선이 \complexity{4HW}개이므로 총 시간복잡도는 \complexity{HW}입니다.
	\end{itemize}
\end{frame}