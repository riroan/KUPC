\include{header}
\usetikzlibrary{arrows.meta,matrix,decorations.pathreplacing}

\title{KUPC 2022 문제 풀이}
\subtitle{Official Solutions}
\author{컴퓨터공학부 김명기, 이승엽, 김태현, 이동훈}
\date{2022년 12월 3일}

\begin{document}
    \setcounter{framenumber}{-1}
    \frame{\titlepage}
        
    \begin{frame} % No title at first slide
        \begin{center}
            \begin{tabular}{cl|l|l}
                \hline
                문제 & & 의도한 난이도 & 출제자 \\
                \hline
                \hline

				\textbf{A} & 가장 큰 정사각형 & \textbf{\color{acbronze}Easy} & \texttt{이동훈} \\
                \textbf{B} & 만쥬의 식사& \textbf{\color{acbronze}Easy} & \texttt{김명기} \\
                \textbf{C} & 비숍 여행 & \textbf{\color{acbronze}Easy} & \texttt{김명기} \\
                \textbf{D} & 시험자리 배정하기 & \textbf{\color{acsilver}Medium} & \texttt{김명기} \\
                \textbf{E} & 즐거운XOR & \textbf{\color{acsilver}Medium} & \texttt{김명기} \\
                \textbf{F} & 킥보드로 등교하기 & \textbf{\color{acsilver}Medium} & \texttt{김태현} \\
                \textbf{G} & 보물찾기 2 & \textbf{\color{acgold}Hard} & \texttt{이동훈} \\
                \textbf{H} & 볼링 아르바이트 & \textbf{\color{acgold}Hard} & \texttt{이승엽} \\
                \textbf{I} & 문자열 게임 & \textbf{\color{acgold}Hard} & \texttt{이승엽} \\
                \textbf{J} & 압도적 XOR수 & \textbf{\color{acgold}Challenging} & \texttt{김태현} \\

                \hline
            \end{tabular}
        \end{center}
    \end{frame}
    \import{solutions/}{A-crying-goose.tex}
    \import{solutions/}{B-eating-manju.tex}
    \import{solutions/}{C-bishop-tour.tex}
    \import{solutions/}{D-set-examseats.tex}
    \import{solutions/}{E-interesting-xor.tex}
    \import{solutions/}{F-kickboard-commutes.tex}
    \import{solutions/}{G-treasure-2.tex}
    \import{solutions/}{H-bowling-part-time.tex}
    \import{solutions/}{I-string-game.tex}
    \import{solutions/}{J-overwhelming-xor-number.tex}

\end{document}
