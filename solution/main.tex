%!TEX program = xelatex
\documentclass[11pt, aspectratio=169]{beamer}
\usefonttheme{professionalfonts}

\usepackage{amsmath}
\usepackage{fontspec}
\usepackage{graphicx}
\usepackage{import}
\usepackage{kotex}
\PassOptionsToPackage{table}{xcolor}
\usepackage{calc}
\usepackage{listings}
\usepackage{indentfirst}
\usepackage{tabularx}
\usepackage{ulem}
\usepackage{multicol}
\usepackage{epigraph}
\usepackage[many]{tcolorbox}

\definecolor{boj}{RGB}{0,118,191}
\definecolor{ucpc-orange}{RGB}{255,153,0}
\definecolor{kupc-gray}{RGB}{177,179,180}
\definecolor{kupc-green}{RGB}{3,107,63}
\definecolor{acgreen}{RGB}{0,159,107}
\definecolor{wared}{RGB}{231,76,60}

\definecolor{acbronze}{RGB}{173,86,0}
\definecolor{acsilver}{RGB}{67,95,122}
\definecolor{acgold}{RGB}{236,154,0}
\definecolor{acplatinum}{RGB}{39,226,164}
\definecolor{acdiamond}{RGB}{0,180,252}
\definecolor{acruby}{RGB}{255,0,98}

\setbeamercolor{title}{fg=black}
\setbeamercolor{frametitle}{fg=ucpc-orange}
\setbeamercolor{structure}{fg=ucpc-orange}

\linespread{1.2}
\everymath{\displaystyle}

\graphicspath{ {./images/} }
\lstset{basicstyle=\footnotesize\ttfamily,breaklines=true}

\newcommand{\translation}[1]{\textsuperscript{#1}}

\setlength\fboxsep{0pt}

\newcommand{\complexity}[1]{$\mathcal{O}\left({#1}\right)$}
\newcommand{\difficulty}[1]{\includegraphics[width=1em,natwidth=1000,natheight=1000]{#1.svg.png}}
\newcommand{\norm}[1]{\left\lVert#1\right\rVert}

\usetheme{Ucpc2020}
\usetikzlibrary{arrows.meta,matrix,decorations.pathreplacing}

\title{KUPC 2022 문제 풀이}
\subtitle{Official Solutions}
\author{컴퓨터공학부 김명기, 이승엽, 김태현, 이동훈}
\date{2022년 12월 3일}

\begin{document}
    \setcounter{framenumber}{-1}
    \frame{\titlepage}
        
    \begin{frame} % No title at first slide
        \begin{center}
            \begin{tabular}{cl|l|l}
                \hline
                문제 & & 의도한 난이도 & 출제자 \\
                \hline
                \hline

				\textbf{A} & 가장 큰 정사각형 & \textbf{\color{acbronze}Easy} & \texttt{이동훈} \\
                \textbf{B} & 만쥬의 식사& \textbf{\color{acbronze}Easy} & \texttt{김명기} \\
                \textbf{C} & 비숍 여행 & \textbf{\color{acbronze}Easy} & \texttt{김명기} \\
                \textbf{D} & 시험자리 배정하기 & \textbf{\color{acsilver}Medium} & \texttt{김명기} \\
                \textbf{E} & 즐거운XOR & \textbf{\color{acsilver}Medium} & \texttt{김명기} \\
                \textbf{F} & 킥보드로 등교하기 & \textbf{\color{acsilver}Medium} & \texttt{김태현} \\
                \textbf{G} & 보물찾기 2 & \textbf{\color{acgold}Hard} & \texttt{이동훈} \\
                \textbf{H} & 볼링 아르바이트 & \textbf{\color{acgold}Hard} & \texttt{이승엽} \\
                \textbf{I} & 문자열 게임 & \textbf{\color{acgold}Hard} & \texttt{이승엽} \\
                \textbf{J} & 압도적 XOR수 & \textbf{\color{acgold}Challenging} & \texttt{김태현} \\

                \hline
            \end{tabular}
        \end{center}
    \end{frame}
    \import{solutions/}{A-crying-goose.tex}
    \import{solutions/}{B-eating-manju.tex}
    \import{solutions/}{C-bishop-tour.tex}
    \import{solutions/}{D-set-examseats.tex}
    \import{solutions/}{E-interesting-xor.tex}
    \import{solutions/}{F-kickboard-commutes.tex}
    \import{solutions/}{G-treasure-2.tex}
    \import{solutions/}{H-bowling-part-time.tex}
    \import{solutions/}{I-string-game.tex}
    \import{solutions/}{J-overwhelming-xor-number.tex}

\end{document}
