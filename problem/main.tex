\documentclass[10pt,a4paper,oneside,dvipsnames]{article}

\usepackage[english,russian]{babel}
\usepackage{olymp}
\usepackage{graphicx}
\usepackage{amsmath}
\usepackage{amssymb}
\usepackage{color} % for colored text
\usepackage{import} % for changing current dir
\usepackage{epigraph}
% \usepackage{daytime} % for displaying version number and date
\usepackage{wrapfig} % for having text alongside pictures
\usepackage{verbatim}
\usepackage{tikz}
\usepackage[figurename=그림]{caption} % caption name
\usepackage{subcaption}
\def\NoInputFileName{}
\def\NoOutputFileName{}

\intentionallyblankpagestrue

\newcommand{\importproblem}[1]{\import{problems/#1/statement/}{./#1.tex}}

\contest
{UCPC 2020}%
{Finals}%
{August 1st, 2020}%

\binoppenalty=10000
\relpenalty=10000
\exhyphenpenalty=10000

\begin{document}

\BuildContestTitle

\raggedbottom

\begin{center}
    \textbf{\Large\textsf{문제 목록}}
\end{center}

문제지에 있는 문제가 총 12문제가 맞는지 확인하시기 바랍니다.

\begin{center}
    \begin{minipage}{0.35\textwidth}
        \begin{itemize}
            \item[\textbf{A}] 전단지 돌리기
            \item[\textbf{B}] 던전 지도
            \item[\textbf{C}] 함수 복원
            \item[\textbf{D}] 소가 길을 건너간 이유 2020
            \item[\textbf{E}] 지도 설치
            \item[\textbf{F}] 애완 트리
            \item[\textbf{G}] 그건 망고가 아니라 고양이예요
            \item[\textbf{H}] 레이저 연구소
            \item[\textbf{I}] 빛의 전사 크리퓨어
            \item[\textbf{J}] 관광 사업
            \item[\textbf{K}] 데이터 제작
            \item[\textbf{L}] 피자 배틀
        \end{itemize}    
    \end{minipage}
\end{center}

모든 문제의 메모리 제한은 1GB로 동일합니다.
\newpage

%\importproblem{sample}
\importproblem{driving-leaflet} % A
\importproblem{dungeon-map} % B
\importproblem{function-restore} % C
%\importproblem{demarcation} % C
\importproblem{cowfly} % D
\importproblem{maps} % E
\importproblem{pet-tree} % F
\importproblem{mango} % G
\importproblem{laser-lab} % H
\importproblem{bitkriii} % I
\importproblem{tour} % J
\importproblem{v-e-f} % K
\importproblem{pizza-battle} % L

\end{document}
