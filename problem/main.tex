\documentclass[10pt,a4paper,oneside,dvipsnames]{article}

\usepackage[english,russian]{babel}
\usepackage{olymp}
\usepackage{graphicx}
\usepackage{amsmath}
\usepackage{amssymb}
\usepackage{color} % for colored text
\usepackage{import} % for changing current dir
\usepackage{epigraph}
% \usepackage{daytime} % for displaying version number and date
\usepackage{wrapfig} % for having text alongside pictures
\usepackage{verbatim}
\usepackage{tikz}
\usepackage[figurename=그림]{caption} % caption name
\usepackage{subcaption}
\def\NoInputFileName{}
\def\NoOutputFileName{}

%\intentionallyblankpagestrue

\newcommand{\importproblem}[1]{\import{problems/#1/statement/}{./#1.tex}}

\contest
{KUPC 2022}%
{December 3th, 2022}%

\binoppenalty=10000
\relpenalty=10000
\exhyphenpenalty=10000

\begin{document}

\BuildContestTitle

%\raggedbottom

\begin{center}
    \textbf{\Large\textsf{문제 목록}}
\end{center}


\begin{center}
    \begin{minipage}{0.35\textwidth}
        \begin{itemize}
            
	\item[\textbf{A}] 슬픈 건구스
	\item[\textbf{B}] 만쥬의 식사
	\item[\textbf{C}] 비숍 여행
	\item[\textbf{D}] 시험자리 배정하기
	\item[\textbf{E}] 즐거운 XOR
	\item[\textbf{F}] 킥보드로 등교하기
	\item[\textbf{G}] 보물 찾기 2
	\item[\textbf{H}] 볼링 아르바이트
	\item[\textbf{I}] 문자열 게임
	\item[\textbf{J}] 압도적 XOR 수


        \end{itemize}    
    \end{minipage}
\end{center}

모든 문제의 메모리 제한은 512MB로 동일합니다.
\newpage

\importproblem{crying-goose}
\importproblem{eating-manju} 
\importproblem{bishop-tour} 
\importproblem{set-examseats} 
\importproblem{interesting-xor} 
\importproblem{kickboard-commute} 
\importproblem{treasure-2} 
\importproblem{bowling-part-time}
\importproblem{string-game} 
\importproblem{overwhelming-xor-number} 


\end{document}
