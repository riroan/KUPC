\begin{problem}{압도적 XOR수}{표준 입력(stdin)}{표준 출력(stdout)}{1\,초}{512\,MB}

양의 정수 $A$에 대해 $A$보다 큰 $2$의 제곱수 중 가장 작은 값을 $P$라고 할 때, $B = A \oplus (P-1)$라고 하자.

이 때, 양의 정수 $K$에 대해 $A \geq B \cdot (2^K-1)$일 경우 $A$를 \textbf{압도적 XOR수}라고 한다.

$N$과 $K$가 주어지면 $1$부터 $N$까지의 정수 중 \textbf{압도적 XOR수}의 개수를 구해보자.

\InputFile
첫 번째 줄에 $N, K$가 공백으로 구분되어 주어진다. $(1 \leq N \leq 10^{15}, 1 \leq K \leq 50)$

\OutputFile
$1$부터 $N$까지의 \textbf{압도적 XOR수}의 개수를 출력한다. 

문제에서 사용되는 모든 정수의 범위는 64비트 정수 이내이다.

\Examples

\begin{example}
\exmp{
243 4
}{%
22
}%
\exmp{
243 7
}{%
7
}%
\exmp{
1000000000000000 2
}{%
437050046578689
}%
\end{example}

\Note
$a$와 $b$의 bitwise XOR인 $a \oplus b$는 2진법으로 표현했을 때 $a$와 $b$의 $i$번째 자리가 같으면 $a \oplus b$의 $i$번째 자리가 $0$이고, 서로 다르면 $1$이 되도록 계산한다.

두 번째 입력에서 \textbf{압도적 XOR수}는 $1, 3, 7, 15, 31, 63, 127$로 총 $7$개이다.

\end{problem}
