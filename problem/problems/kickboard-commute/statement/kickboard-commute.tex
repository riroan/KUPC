\begin{problem}{킥보드로 통학하기}{표준 입력(stdin)}{표준 출력(stdout)}{2\,초}{256\,MB}

건덕이는 이번 학기 수강신청을 실패해 1교시 수업을 잔뜩 듣게 되었다! 건덕이의 등교시간은 직장인의 출근시간과 겹치는 시간대였기 때문에, 지하철이 아닌 킥보드를 구매해서 등교하기로 한다.

킥보드는 이동하는 거리만큼 배터리가 소모되며, 배터리를 모두 사용하면 반드시 충전해야 한다.

건덕이의 집과 학교는 각각 $0$, $L$에 위치해 있으며, 등굣길에는 총 $N$개의 킥보드 충전소가 순서대로 위치해 있다. 충전하느라 시간을 낭비한다면 지각할 게 뻔하기 때문에, 건덕이는 등교 중에 최대 $K$번 충전하기로 했다. 충전소에 방문하면 킥보드의 배터리가 가득 찬다.

킥보드의 가격과 배터리 용량은 비례하며, 건덕이는 집에서 킥보드를 모두 충전한 상태로 집을 나선다. 

건덕이는 조건을 만족하는 킥보드 중에서도 가장 싼 킥보드를 구매하고자 한다. 건덕이가 구매할 킥보드의 배터리 용량을 구해보자. 

\InputFile
첫 번째 줄에 학교까지의 거리, 킥보드 충전소의 개수, 최대 충전 횟수를 나타내는 세 정수 $L, M, K$가 공백으로 구분되어 주어진다.

두 번째 줄에 $i$번째 충전소의 위치를 나타내는 $N$개의 정수 $A_i$가 공백으로 구분되어 주어진다.

\begin{itemize}
 \item $5 ≤ L ≤ 200\,000$

 \item $3 ≤ N ≤ \min(L - 1, 100\,000)$

 \item $0 ≤ K ≤ N$

 \item $1 ≤ A_i < L$

 \item $1 ≤ i < N$인 모든 $i$에 대해서 $A_i < A_{i+1}$
\end{itemize}


\OutputFile
건덕이가 구매해야 할 킥보드의 배터리 용량을 출력한다.

\Examples

\begin{example}
\exmp{
10 3 2
1 3 6
}{%
4
}%
\exmp{
5 4 2
1 2 3 4
}{%
2
}%
\end{example}

\end{problem}
