\begin{problem}{비숍 여행}{표준 입력(stdin)}{표준 출력(stdout)}{1\,초}{512\,MB}


크기가 무한한 체스판에 비숍의 위치가 주어진다. 체스판 위에 $N$개의 동전이 놓여져 있으며 비숍이 동전이 놓인 칸에 도달하면 동전을 획득할 수 있다.

비숍을 임의의 횟수만큼 이동해서 획득할 수 있는 동전의 최대 개수를 구해보자.

단, 비숍은 현재 위치가 $(x,y)$라면 $(x-1, y-1), (x-1, y+1), (x+1, y-1), (x+1, y+1)$ 중 한 곳으로 이동할 수 있다.

\InputFile
첫 번째 줄에 비숍이 놓여져 있는 좌표를 나타내는 정수 $x,y$가 공백으로 구분되어 주어진다. $(1 \le x, y \le 10^9)$

두 번째 줄에 동전의 개수 $N$이 주어진다. $(1 \le N \le 200\,000)$

세 번째 줄부터 $N$개의 줄에 걸쳐 동전의 좌표를 나타내는 정수 $a_x, a_y$가 공백으로 구분되어 주어진다. $(1 \le a_x, a_y \le 10^9)$

비숍의 위치와 동전의 위치가 중복되는 경우가 없고 위치가 중복된 동전은 존재하지 않는다.

\OutputFile
비숍을 임의의 횟수만큼 이동해서 획득할 수 있는 동전의 최대 개수를 출력한다.

\Examples

\begin{example}
\exmp{
1 1
5
2 2
3 1
2 4
1 2
3 4
}{%
3
}%
\exmp{
1 2
3
1 1
2 2
3 3
}{%
0
}%
\end{example}

\end{problem}
