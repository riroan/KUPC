\begin{problem}{함수 복원}{표준 입력(stdin)}{표준 출력(stdout)}{1.5\,초}{1024\,MB}

자연수 $N$이 주어질 때, $1$ 이상 $N$ 이하의 자연수 $i$에 대해 정의되는 함수 $f$가 있다. 모든 $i$에 대해 $f\left(i\right)$ 또한 $1$ 이상 $N$ 이하의 자연수다.

$f$에 대한 함수 그래프를 정점 $i$에서 정점 $f\left(i\right)$로 향하는 단방향 간선들로 이루어진 그래프라 부르자. 이 그래프는 $f$에 따라 유일하게 결정됨을 알 수 있다.

우리는 함수 $f$를 알지 못하지만, $f$에 대한 함수 그래프에서 모든 정점에 대한 도달 가능성 정보를 가지고 있다. 정점 $u$에서 정점 $v$에 도달 가능하다는 것은 $0$개 이상의 간선을 통해 정점 $u$에서 정점 $v$로 갈 수 있다는 뜻이다.

함수 $f$로 가능한 경우의 수를 구하여라.

\InputFile
첫 줄에 $N$이 주어진다. ($1 \leq N \leq 500$)

그 후, $N$개의 줄에 걸쳐 $f$에 대한 함수 그래프에서 도달 가능성 정보가 공백으로 구분되어 주어진다. $i$번째 줄의 $j$번째 수는 $i$에서 $j$에 도달 가능하면 $1$, 그렇지 않으면 $0$이다. 가능한 함수 $f$가 존재하지 않는 경우는 주어지지 않는다.

\OutputFile
함수 $f$로 가능한 경우의 수를 $10^9 + 7$로 나눈 나머지를 출력한다. 

\Examples

\begin{example}
\exmp{
6
1 1 1 1 0 0
0 1 1 1 0 0
0 1 1 1 0 0
0 1 1 1 0 0
1 1 1 1 1 0
0 0 0 0 0 1
}{%
6
}%
\exmp{
9
1 0 1 1 0 0 0 0 0
1 1 1 1 0 0 0 0 0
1 0 1 1 0 0 0 0 0
1 0 1 1 0 0 0 0 0
0 0 0 0 1 0 0 0 0
0 0 0 0 1 1 0 0 1
1 1 1 1 0 0 1 0 0
0 0 0 0 1 1 0 1 1
0 0 0 0 1 0 0 0 1
}{%
6
}%
\end{example}

\end{problem}

