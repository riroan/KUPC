\begin{problem}{길이 문자열}{표준 입력(stdin)}{표준 출력(stdout)}{1\,초}{1024\,MB}
    
길이 문자열은 숫자 0--9와 하이픈(`-')으로만 이루어진 문자열 중 다음 조건을 만족하는 것을 가리킵니다.

\begin{itemize}
    \item `-'이 2개 이상 연속해서 등장하지 않습니다.
    \item 문자열의 첫 문자는 `0'이 아닙니다.
    \item 문자열의 마지막 문자는 `-'이 아닙니다.
    \item `-'의 다음 문자로 `0'이 등장하지 않습니다.
    \item 문자열의 숫자로만 이루어진 접미사 중 가장 긴 것을 10진법의 수로 해석하면 문자열의 길이와 같습니다. 이 때 그러한 접미사가 빈 문자열이면 0으로 해석합니다.
    \item 문자열에 `-'이 등장한다면 문자열의 처음부터 가장 마지막에 등장하는 `-' 앞까지의 부분 문자열이 길이 문자열입니다.
\end{itemize}

임의의 음이 아닌 정수 $n$에 대해 길이가 $n$인 길이 문자열은 유일하게 한 개 존재합니다. 다음은 각각 길이 5, 8, 13인 길이 문자열의 예시입니다.

\texttt{
1-3-5
-2-4-6-8
1-3-5-7-10-13
}

자연수 $a$, $b$가 주어졌을 때 길이가 $a^b$인 길이 문자열을 찾아봅시다.


\InputFile
첫째줄에 테스트 케이스의 개수 $T$가 주어집니다. ($1 \leq T \leq 100,000$)

각 테스트 케이스마다 두 개의 자연수 $a$, $b$가 공백으로 구분되어 한 줄에 주어집니다. ($1 \leq a,\, b \leq 10^7$, $a^b \leq 10^{10^6}$)


\OutputFile
각 테스트 케이스마다 길이 $a^b$인 길이 문자열을 출력합니다. 이때 $a^b \geq 22$이면 문자열의 맨 앞의 9글자와 맨 뒤 9글자만 출력 예시와 같은 형식으로 출력합니다.

\Examples

\begin{example}
v\exmp{
2 1
1 2
}{%
1
}%
\exmp{
3 3
1 2 3
}{%
0
}%
\end{example}

\end{problem}
